During the semester, we have implemented a full compiler for Tool written in Scala. It outputs Java byte code which can consequently run on the JVM.

Throughout this period, we had the occasion to write quite a number of Tool programs, which allowed us to notice that it could be tedious to debug Tool code. That's why we chose to develop a debugger for Tool.

\subsection{Features}

\begin{enumerate}
\item Load Tool source files, or develop directly in the web IDE
\item Set breakpoints in the code
\item Step-by-step execution: step over lines, step into / out function calls
\item See the value of every variable
\item See the current call stack
\item integrated code editor and compiler
\end{enumerate}

We chose to implement our debugger as a single web page such that it is cross platform, easy to use (no binary to download) and can still be used offline. Moreover, it allows us to leverage the browser's garbage collector to implement easily the debugging VM (see "Virtual machine implementation")

% [Describe in a few words what you did in the first part of the compiler project
%(the non-optional labs)]Lorem ipsum dolor sit amet, consectetur adipisicing elit, sed do eiusmod
%tempor incididunt ut labore et dolore magna aliqua. Ut enim ad minim veniam,
%quis nostrud exercitation ullamco laboris nisi ut aliquip ex ea commodo

%[and briefly say what problem you want to solve with
%your extension.] In the process of writing our compiler we faced major issues %for writing tool programs. Without a way to debug our tool programs, we can't %know for sure wether a given bug is due to our compiler or our program. We %therefore found it could be useful not only for us but also for future %iterations of this course to have a tool language debugger. 

%This extension does not involve many modifications of the previously achieved %work but requires a bit of additionnal software. 



%This section should convince us that you have a clear picture of the general
%architecture of your compiler and that you understand how your extension fits
%in it.
